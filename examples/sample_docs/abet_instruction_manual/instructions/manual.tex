\documentclass{article}

\date{}

\usepackage{color}   %May be necessary if you want to color links
\usepackage{hyperref}
\hypersetup{
    colorlinks=true, %set true if you want colored links
    linktoc=all,     %set to all if you want both sections and subsections linked
    linkcolor=blue,  %choose some color if you want links to stand out
}

\usepackage[margin=1in]{geometry}

\title{ABET Internal Network Instruction Manual}

\begin{document}

\maketitle

%\tableofcontents

\section{Connecting to Wi-Fi}

The wireless access point for the ABET network is located in the Dean's
Conference Room. To connect to the ABET network Wi-Fi, connect using:
\\\\
\indent Network Name: \textbf{ABET\_WiFi}, no password is required.
\\\\
Once connected, you should be able to access the internet as well as the
internal ABET fileshare.

\section{Accessing the Network Fileshare}

In order to make sharing documents as seamless as possible, a local fileshare
server is setup on the ABET network where all users on the network can access
and upload documents to a shared folder. Users can upload/download documents
to/from the fileshare folder from any device via any web browser (no additional
software is needed).
\\\\
To access the fileshare:
\begin{enumerate}
  \item{Connect your device to ABET\_WiFi}
%   \\
%   This can be done via connecting the device to ABET\_WiFi or using either of
%   the desktop workstations in the Dean's conference room.

  \item{Go to \url{http://abetshare/}}
    \\
    On your network-connected device, open a web browser and type
    \url{http://abetshare/} in the address bar and press [ENTER]. This should
    direct you to the fileshare webpage.

  \item{Using the fileshare}
    \\
    The fileshare webpage will display an ABET\_SHARE folder, this is where the
    fileshare documents are contained. Click the ABET\_SHARE folder to open it.
    The fileshare webpage behaves similarly to a desktop file explorer, you can
    navigate through folders and open files by clicking them. The menu on the
    left side of the webpage has additional options such as upload, delete, and
    create new folder.

\end{enumerate}

\section{Using the Desktop Workstations}

There are two, complimentary Windows workstations located in the Dean's
conference room that are connected to the ABET network. \textbf{No credentials are
required to sign in}.
\\\\
There is a special \textbf{Desktop\textbackslash ABET\_SHARE} folder on each
workstation that is automatically synced to the ABET fileshare. This means
that both workstations will automatically update the contents of
this folder with the documents in the network fileshare so you can simply treat the
fileshare like a regular system folder. Adding files to the ABET\_SHARE
folder uploads them to the fileshare, files uploaded to the fileshare are
automatically downloaded and appear in the ABET\_SHARE folder.

\end{document}
