% this is a comment in TeX, comments are messages used to clarify code and
% comments are ignored by the compiler

% the document class is the type of document that you are creating different
% document classes have different ways to handle titles, paragraphs, etc. to see
% all the different types of document classes, go to
% https://ctan.org/topic/class
\documentclass{refman}

% there are TONS of packages for LaTeX that you can install and then implement
% in your own TeX files by using the usepackage tag
% the graphicx package allows us to include graphics into our document using the
% includegrahics function
\usepackage{graphicx}

% these lines set the title and author information for this document
% special characters have a specific name in LaTeX and you specify them by using
% the backslash. The \LaTeX symbol is actually a special character that prints
% out the LaTeX logo in that fancy font
\title{Hello {\LaTeX}}
\author{John McAvoy}

% in TeX, tags that can be specified on one line look like the title and author
% tags, start with a backslash and then the tag name: e.g. \tag

% tags that span multiple lines and have other things nested inside them follow
% a \begin{tag} ... \end{tag} format such as the document tag below
\begin{document}

% this is the simple command that will generate the title and author for you
\maketitle
% LaTeX will also generate table of contents based on your section titles
\tableofcontents

% printing text is a simple as typing it out
Hello World!\\

% to specify new lines use two backslashes
\textit{Hello there! I am an italic font}\\
\textbf{I am a bold one}\\
\underline{I am an underlined font}\\
\emph{I use emph for emphasis, EMPHASIS!}\\

\begin{section}{Sections}
  Sections are useful ways to break-up your paper

  \begin{subsection}{Sub-Sections}
    Subsections specify even more detail

    \begin{subsubsection}{Sub-Sub-Sections}
      Subsubsections are cool too!
    \end{subsubsection}

  \end{subsection}

\end{section}

\begin{section}{Making lists}

  \begin{subsection}{Enumerations}
    \begin{enumerate}
      \item{I am number 1!}
      \item{1st is the worst, 2nd is the best}
      \item{Blah}
      \item{Blah blah}
    \end{enumerate}
  \end{subsection}

  \begin{subsection}{Itemizations}
    \begin{itemize}
      \item{I am number 1!}
      \item{No one care, there is no order in Itemizations}
      \item{Blah}
      \item{Blah blah}
    \end{itemize}
  \end{subsection}

  \begin{subsection}{Tables}
    \begin{tabular}{ l c r }
      1 & 2 & 3 \\
      4 & 5 & 6 \\
      7 & 8 & 9 \\
    \end{tabular}
  \end{subsection}

  \begin{subsection}{Images}
    % you can specify different parameters inside the square braces of a TeX
    % function, in this case, I am telling includegraphics to include the
    % panda_icon.png image and set its width in the document to 40px
    \includegraphics[width=40px]{./panda_icon.png}
  \end{subsection}

  \begin{subsection}{Equations}
    One of the biggest advantages that {\LaTeX} has over word processors is how
    easy it is to make equations
    % inline-equations can be written between dollar signs
    $ C = 2 \pi r$

    % or you can use begin...end tags
    \begin{equation}
      E = m c^2
    \end{equation}

    % super scripting
    $ c^2 = a^2 + b^2$ \\
    $ c = \sqrt{a^2 + b^2}$ \\
    $ c = \sec(\frac{a}{b})$ \\

    % sub scripting
    $m_1 v_1 + m_2 v_2 = (m_1 + m_2) v_f$ \\

    % this still is not scratching the surface
%    $$ f^{'}(x) = \lim_{h \to 0} \frac{ f( x + h ) - f( x ) }{h}$$
  \end{subsection}

\end{section}

% the closing to the document tag
\end{document}
